\chapter*{Definition List}
\begin{description}

\item[Blitz Attack] \hfill
\begin{description}
	\item To aggressively attack a fire from the exterior with a large diameter (2-1/2 inch [65 mm] or larger) fire stream.
\end{description}

\item[Defensive Attack] \hfill
\begin{description}
	\item (Defensive Strategy) Overall plan for incident control established by the Incident Commander (IC) that involves protection of exposures, as opposed to aggressive, offensive intervention. 
	\item Exterior fire suppression operations directed at protecting exposures. 
	\item The mode of manual fire control in which the only fire suppression activities taken are limited to those required to keep a fire from extending from one area to another.
\end{description}

\item[Direct Attack] \hfill
\begin{description}
	\item Attack method that involves the discharge of water or a foam stream directly onto the burning fuel.
	\item Firefighting operations involving the application of extinguishing agents directly onto the burning fuel.
	\item Fire-fighting operations involving the application of extinguishing agents directly onto the burning fuel.
\end{description}

\item[Exterior Attack] \hfill\\

\item[Fire] \hfill
\begin{description}
	\item A rapid oxidation process, which is a chemical reaction resulting in the evolution of light and heat in carrying intensities.
	\item A rapid, persistent chemical reaction that releases both heat and light.
	\item A rapid oxidation process, which is a chemical reaction resulting in the evolution of light and heat in varying intensities.
\end{description}

\item[Fire Attack] \hfill\\

\item[Flashover] \hfill
\begin{description}
	\item A rapid transition from the growth stage to the fully developed stage.
	\item A transition stage in the development of a compartment fire in which surfaces exposed to thermal radiation reach ignition temperature more or less simultaneously and fire spreads rapidly throughout the space, resulting in full room involvement or total involvement of the compartment or enclosed space.
	\item A transition phase in the development of a compartment fire in which surfaces exposed to thermal radiation reach ignition temperature more or less simultaneously and fire spreads rapidly throughout the space, resulting in full room involvement or total involvement of the compartment or enclosed space.
\end{description}

\item[Flow Path] \hfill
\begin{description}
	\item Composed of at least one inlet opening, one exhaust opening, and the connecting volume between the openings.  The direction of the flow is determined by difference in pressure.  Heat and smoke in a high-pressure area will flow toward areas of lower pressure.  
	\item The volume in a structure between an inlet and an outlet that allows the movement of heat and smoke from the higher pressure within the fire area toward the lower pressure areas accessible via doorways, halls, stairs, and window openings.
\end{description}

\item[Fuel Limited Fire] \hfill
\begin{description}
	\item (Fuel Controlled) A fire with adequate oxygen in which the heat release rate and growth rate are determined by the characteristics of the fuel, such as quantity and geometry.
	\item A fire that has sufficient oxygen for fire growth but has a limited amount of furl available for burning.
	\item A fire that has a heat release rate that is controlled by the material burning.
\end{description}

\item[Half Bale] \hfill
\begin{description}
	\item Using the broken stream of the nozzle to get water into the fire room above grade level. This is a secondary water application method for the exterior portion of transitional attack.
\end{description}

\item[Indirect Attack] \hfill
\begin{description}
	\item Form of fire attack that involves directing fire streams toward the ceiling of a compartment in order to generate a large amount of steam in order to cool the compartment.  Converting the water to steam displaces oxygen, absorbs the heat of the fire, and cools the hot has layer sufficiently for firefighters to safely enter and make a direct attack on the fire.  
	\item Firefighting operations involving the application of extinguishing agents to reduce the build-up of heat released from a fire without applying the agent directly onto the burning fuel.
	\item Fire-fighting operations involving the application of extinguishing agents to reduce the buildup of heat released from a fire without applying the agent directly onto the burning fuel.
\end{description}

\item[Interior Attack] \hfill
\begin{description}
	\item The assignment of a team of fire fighters to enter a structure and attempt fire suppression.
\end{description}

\item[Jargon] \hfill
\begin{description}
	\item The specialized or technical language of a trade, profession, or similar group.
\end{description}

\item[Knock Back] \hfill
\begin{description}
	\item A state of partial fire extinguishment wihch will allow for fire regrow in a short period of time without additional intervention.
\end{description}

\item[Knock Down] \hfill
\begin{description}
	\item A state of partial fire extinguishment that is close to full extinguishment and where regrowth is unlikely.
\end{description}

\item[Offensive Attack] \hfill
\begin{description}
	\item (Offensive Strategy) Overall plan for incident control established by the Incident Commander (IC) in which responders take aggressive, direct action on the material, container, or process equipment involved in an incident.
	\item An advance into the fire building by firefighters with hose lines or other extinguishing agents that are intended to overpower the fire.
	\item The mode of manual fire control in which manual fire suppression activities are concentrated on reducing the size of a fire to accomplish extinguishment.
\end{description}

\item[Size-up] \hfill
\begin{description}
	\item Ongoing evaluation of influential factors at the scene of an incident.
	\item The observation and evaluation of existing factors that are used to develop objectives, strategy, and tactics for fire suppreassion.
	\item The observation and evaluation of existing factors in order to develop objectives, strategies, and tactics for fire suppression.
\end{description}
 
\item[Transitional Attack] \hfill
\begin{description}
	\item A Fire Attack where the application of water starts on the exterior and transitions to an interior application. 
	\item An offensive fire attack initiated by an exterior, indirect hand line operation into the fire compartment to initiate cooling while transitioning into interior direct fire attack in coordination with ventilation operations. 
\end{description}

\item[Ventilation Limited Fire] \hfill
\begin{description}
	\item (Ventilation Controlled) A fire with limited ventilation in which the heat-release rate or growth is limited by the amount of oxygen available to the fire.
	\item A fire in an enclosed building that is restricted because there is insufficient oxygen available for the fire to burn as rapidly as it would with an unlimited supply of oxygen. 
	\item A fire where every object in the fire compartment is fully involved in fire and the heat release rate depends on the airflow through the openings to the fire compartment.
\end{description}


\item[Whip] \hfill
\begin{description}
	\item An addition form of exterior water application used in transitional attack to gain better hold time on the fire room. Typically the nozzle is positioned inside the window of the room and opened to a half bale or fog stream and moved in a rapid circular below the plane of the window sill for a period of time. 
\end{description}

\end{description}