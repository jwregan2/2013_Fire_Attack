\chapter*{Definition List}
\addcontentsline{toc}{chapter}{Definition List}

\begin{description}

\item[Atmospheric Pressure] \hfill
\begin{description}[leftmargin=!]
	\item The pressure of the weight of air on the surface of the earth, approximately 14.7~pounds~per~square~inch (psia) (101~kPa absolute) at sea level.
\end{description}

\item[Bi-directional vent] \hfill
\begin{description}[leftmargin=!]
	\item A building opening that serves as both as an intake and exhaust vent of a flow path at the same time. 
\end{description}

\item[Broken Stream] \hfill
\begin{description}[leftmargin=!]
	\item A stream of water that has been broken into coarsely divided drops which is usually created by the rapid movement of a nozzle. 
\end{description}

\item[Burning Rate] \hfill
\begin{description}[leftmargin=!]
	\item See \hyperlink{HRR}{Heat Release Rate (HRR)}.
\end{description}

\item[Calorie] \hfill
\begin{description}[leftmargin=!]
	\item The amount of heat necessary to raise 1 gram of water 1~$^\circ$C at the pressure of 1 atmosphere and temperature of 15~$^\circ$C; a calorie is 4.184~joules, and there are 252.15~calories in a British thermal unit (Btu).
\end{description}

\item[Combustion] \hfill
\begin{description}[leftmargin=!] 
	\item A chemical process of oxidation that occurs at a rate fast enough to produce heat and usually light in the form of either a glow or flame.
\end{description}

\hypertarget{cp}{\item[Combustion Products]} \hfill
\begin{description}[leftmargin=!]
	\item The heat, gases, volatilized liquids and solids, particulate matter, and ash generated by combustion.
\end{description}

\item[Compartment] \hfill
\begin{description}[leftmargin=!]
	\item A room or are that is subdivided from other areas of the structure.
\end{description}

\item[Conduction] \hfill
\begin{description}[leftmargin=!]
	\item Heat transfer to another body or within a body by direct contact.
\end{description}

\item[Convection] \hfill
\begin{description}[leftmargin=!]
	\item Heat transfer by circulation within a medium such as a gas or a liquid. 
\end{description}

\item[Direct Attack] \hfill
\begin{description}[leftmargin=!]
	\item Attack method that involves the discharge of water or a foam stream directly onto the burning fuel.
	\item Firefighting operations involving the application of extinguishing agents directly onto the burning fuel.
	\item Fire-fighting operations involving the application of extinguishing agents directly onto the burning fuel.
\end{description}

\item[Differential Pressure] \hfill
\begin{description}[leftmargin=!]
	\item The difference between pressures at different points along a flow path.  The pressure difference creates the flow of gases or fluids from an area of higher pressure to an area of lower pressure.
\end{description}

\item[Extinguishment] \hfill
\begin{description}[leftmargin=!]
	\item A state of non fire growth.
	\item To completely stop the combustion process.
\end{description}

\item[Fire] \hfill
\begin{description}[leftmargin=!]
	\item A rapid oxidation process, which is a gas phase chemical reaction resulting in the evolution of light and heat in varying intensities.
\end{description}

\item[Fire Attack] \hfill
\begin{description}[leftmargin=!]
	\item See Fire Suppression
\end{description}

\item[Fire Control]  \hfill
\begin{description}[leftmargin=!]
	\item Limiting the size of the fire so as to decrease the heat release rate and preventing fire spread to adjacent combustibles, while reducing fire gas temperatures.     
\end{description}

\item[Fire Dynamics] \hfill
\begin{description}[leftmargin=!]
	\item The detailed study of how chemistry, fire science, and the engineering disciplines of fluid mechanics and heat transfer interact to influence fire behavior.
\end{description}

\item[Fire Gases] \hfill
\begin{description}[leftmargin=!]
	\item Smoke, products of combustion and pyrolyzates, and gaseous fuels that are present and could ignite and increase the size and intensity of the fire. 
\end{description}

\item[Fire Science] \hfill
\begin{description}[leftmargin=!]
	\item The body of knowledge concerning the study of fire and related subjects (such as combustion, flame, products of combustion, heat release, heat transfer, fire and explosion chemistry, fire and explosion dynamics, thermodynamics, kinetics, fluid mechanics, fire safety) and their interaction with people, structures, and the environment.
\end{description}

\item[Fire Spread] \hfill
\begin{description}[leftmargin=!]
	\item The movement of fire from one place to another.
\end{description}

\item[Fire Suppression] \hfill
\begin{description}[leftmargin=!]
	\item The activities involved in controlling and extinguishing fire.
\end{description}

\item[Flame] \hfill
\begin{description}[leftmargin=!]
	\item A body or stream of gaseous material involved in the combustion process and emitting radiant energy at specific wavelength bands determined by the combustion chemistry of the fuel. In most cases, some portion of the emitted radiant energy is visible to the human eye.
\end{description}

\hypertarget{flameover}{\item[Flameover]} \hfill
\begin{description}[leftmargin=!]
	\item The condition where unburned fuel  from a fire has accumulated in the ceiling layer to a sufficient concentration (i.e., at or above the lower flammable limit) that it ignites and burns; can occur without ignition of, or prior to, the ignition of other fuels separate from the origin.
\end{description}

\item[Flammable] \hfill
\begin{description}[leftmargin=!]
	\item Capable of burning with a flame.
\end{description}

\item[Flashover] \hfill
\begin{description}[leftmargin=!]
	\item A rapid transition from the growth stage to the fully developed stage.
	\item A transition stage in the development of a compartment fire in which surfaces exposed to thermal radiation reach ignition temperature more or less simultaneously and fire spreads rapidly throughout the space, resulting in full room involvement or total involvement of the compartment or enclosed space.
	\item A transition phase in the development of a compartment fire in which surfaces exposed to thermal radiation reach ignition temperature more or less simultaneously and fire spreads rapidly throughout the space, resulting in near full room involvement .
\end{description}

\item[Flow Path] \hfill
\begin{description}[leftmargin=!]
	\item Composed of at least one inlet opening, one exhaust opening, and the connecting volume between the openings.  The direction of the flow is determined by difference in pressure.  Heat and smoke in a high-pressure area will flow toward areas of lower pressure.  
	\item The volume in a structure between an inlet and an outlet that allows the movement of heat and smoke from the higher pressure within the fire area toward the lower pressure areas accessible via doorways, halls, stairs, and window openings.
	\item The area(s) within a structure where heat, smoke and air flows from an area of higher pressure to lower pressure.  It is composed of at least one intake vent, one exhaust vent and the connecting volume between the vents.  
\end{description}

\item[Fuel] \hfill
\begin{description}[leftmargin=!]
	\item A material that will maintain combustion under specified environmental conditions.
\end{description}

\item[Fuel Limited Fire] \hfill
\begin{description}[leftmargin=!]
	\item (Fuel Controlled) A fire with adequate oxygen in which the heat release rate and growth rate are determined by the characteristics of the fuel, such as quantity and geometry.
	\item A fire that has sufficient oxygen for fire growth but has a limited amount of furl available for burning.
	\item A fire that has a heat release rate that is controlled by the material burning.
	\item A fire in which the heat release rate and growth rate are controlled by the characteristics of the fuel, such as combustibility, quantity and geometry, and in which adequate air for combustion is available.   
\end{description}

\item[Fuel Load] \hfill
\begin{description}[leftmargin=!]
	\item The total quantity of combustible contents of a building, space, or fire area, including interior finish and trim, prior to ignition.
\end{description}

\item[Gas] \hfill
\begin{description}[leftmargin=!]
	\item The physical state of a substance that has no shape or volume of its own and will expand to take the shape and volume of the container or enclosure it occupies.
\end{description}

\item[Half Bale] \hfill
\begin{description}[leftmargin=!]
	\item Using the broken stream of the nozzle to get water into the fire room above grade level. This is a secondary water application method for the exterior portion of transitional attack.
\end{description}

\item[Heat] \hfill
\begin{description}[leftmargin=!]
	\item A form of energy characterized by vibration of molecules and capable of initiating and supporting chemical changes and changes of state.
\end{description}

\item[Heat Flux] \hfill
\begin{description}[leftmargin=!]
	\item The measure of the rate of heat transfer to a surface, expressed in kilowatts/m$^2$, kilojoules/m$^2$*sec, or Btu/ft$^2$*sec.
\end{description}

\hypertarget{HRR}{\item[Heat Release Rate (HRR)]} \hfill
\begin{description}[leftmargin=!]
	\item The rate at which heat energy is generated by burning.
\end{description}

\item[Indirect Attack] \hfill
\begin{description}[leftmargin=!]
	\item Form of fire attack that involves directing fire streams toward the ceiling of a compartment in order to generate a large amount of steam in order to cool the compartment.  Converting the water to steam displaces oxygen, absorbs the heat of the fire, and cools the hot has layer sufficiently for firefighters to safely enter and make a direct attack on the fire.  
	\item Firefighting operations involving the application of extinguishing agents to reduce the build-up of heat released from a fire without applying the agent directly onto the burning fuel.
	\item Fire-fighting operations involving the application of extinguishing agents to reduce the buildup of heat released from a fire without applying the agent directly onto the burning fuel.
\end{description}

\item[Interior Attack] \hfill
\begin{description}[leftmargin=!]
	\item The assignment of a team of fire fighters to enter a structure and attempt fire suppression.
\end{description}

\item[Ignition] \hfill
\begin{description}[leftmargin=!]
	\item The process of initiating self-sustained combustion.
\end{description}

\item[Immediate Dangerous  to Life and Health (IDLH)] \hfill
\begin{description}[leftmargin=!]
	\item Any condition that would pose an immediate or delayed threat to life or irreversible adverse health effects.
\end{description}

\item[Jargon] \hfill
\begin{description}[leftmargin=!]
	\item The specialized or technical language of a trade, profession, or similar group.
\end{description}

\item[Joule]  \hfill
\begin{description}[leftmargin=!]
	\item The preferred SI unit of heat, energy, or work. A joule is the heat produced when one ampere is passed through a resistance of one ohm for one second, or it is the work required to move a distance of one meter against a force of one newton. There are 4.184~joules in a calorie, and 1055~joules in a British thermal unit (Btu). A watt is a joule/second. 
\end{description}

\item [Kilowatt] \hfill
\begin{description}[leftmargin=!]
	\item A measurement of energy release rate.  A kilowatt is 1000~watts.  A watt is a joule/second.
\end{description}

\item[Knock Back] \hfill
\begin{description}[leftmargin=!]
	\item A state of partial fire extinguishment wihch will allow for fire regrow in a short period of time without additional intervention.
\end{description}

\item[Knock Down] \hfill
\begin{description}[leftmargin=!]
	\item A state of partial fire extinguishment that is close to full extinguishment and where regrowth is unlikely.
\end{description}

\item [Neutral Plane] \hfill
\begin{description}[leftmargin=!]
	\item The interface or level of zero differential pressure at a compartment vent, such as a door or window, between the higher pressure hot gas flowing out of a fire compartment and the lower pressure cooler air flowing into the compartment.  
\end{description}

\item [Oxygen Deficiency] \hfill
\begin{description}[leftmargin=!]
	\item Insufficiency of oxygen to support combustion. 
\end{description}

\item [Positive Pressure Ventilation] \hfill
\begin{description}[leftmargin=!]
	\item The utilization of powered blowers or fans , post-fire control, to exhaust heat and smoke from the fire area.   
\end{description}

\item [Pressure] \hfill
\begin{description}[leftmargin=!]
	\item Pressure is a measure of force per unit area exerted on a surface at 90 degrees to that surface. Values for pressure may be given in pounds per square inch (psi) or Pascals (Pa) The earth is surrounded by an atmosphere made up of approximately 78\% nitrogen, 21\% oxygen and 1\% of other gases.  The weight of these gases on the earth creates a force of 14.7~pounds per square inch (psi) at sea level.  This is referred to as Atmospheric pressure. Pressure in the fire service is typically referenced in the units of pounds per square inch or PSI, as this is the standard pressure unit for many of the pump panel gauges on an engine.  The pressure shown on the pump panel gauge is actually measured relative to the atmospheric pressure.  In other words, 50~psi is really 50~psi over the atmospheric pressure.  This type of pressure measurement is referred to as psi gauge or psig.   The pressure developed by the fire or by a fan is the measured pressure over and above the atmospheric pressure.   Fires create pressure that push smoke and gases throughout a room or structure. The pressures are very small, on the order of one thousandth of a psi.  Therefore it is best to use a different unit for measuring pressure. This unit is called a Pascal.  When it is written, it is abbreviated as Pa.  101,325~Pa equals 14.7~psi.  Or 1~Pa equals 0.00015~psi.  
\end{description}

\item [Products of Combustion] \hfill
\begin{description}[leftmargin=!]
	\item See \hyperlink{cp}{Combustion Products}.
\end{description}

\item [Pyrolysis] \hfill
\begin{description}[leftmargin=!]
	\item A process in which material is decomposed, or broken down, into simpler molecular compounds by the effects of heat alone; pyrolysis often precedes combustion.
\end{description}

\item [Radiant Heat] \hfill
\begin{description}[leftmargin=!]
	\item Heat energy carried by electromagnetic waves that are longer than light waves and shorter than radio waves; radiant heat (electromagnetic radiation) increases the sensible temperature of any substance capable of absorbing the radiation, especially solid and opaque objects.
\end{description}

\item [Radiation] \hfill
\begin{description}[leftmargin=!]
	\item Heat transfer by way of electromagnetic energy.
\end{description}

\item [Rate of Heat Release] \hfill
\begin{description}[leftmargin=!]
	\item See \hyperlink{HRR}{Heat Release Rate (HRR)}.
\end{description}

\item [Risk] \hfill
\begin{description}[leftmargin=!]
	\item The degree of peril; the possible harm that might occur that is represented by the statistical probability or quantitative estimate of the frequency or severity of injury or loss.
\end{description}

\item [Rollover] \hfill
\begin{description}[leftmargin=!]
	\item See \hyperlink{flameover}{Flameover}.
\end{description}

\item[Size-up] \hfill
\begin{description}[leftmargin=!]
	\item Ongoing evaluation of influential factors at the scene of an incident.
	\item The observation and evaluation of existing factors that are used to develop objectives, strategy, and tactics for fire suppreassion.
	\item The observation and evaluation of existing factors in order to develop objectives, strategies, and tactics for fire suppression.
\end{description}

\item [Smoke] \hfill
\begin{description}[leftmargin=!]
	\item The airborne solid and liquid particulates and gases evolved when a material undergoes pyrolysis or combustion, together with the quantity of air that is entrained or otherwise mixed into the mass.
\end{description}

\item [Soot] \hfill
\begin{description}[leftmargin=!]
	\item Black particles of carbon produced in a flame.
\end{description}

\item [Steam conversion] \hfill
\begin{description}[leftmargin=!]
	\item The physical event where water is delivered to the heat of a fire and the water is converted from a liquid to a vapor in the form of steam. 
\end{description}

\item [Temperature] \hfill
\begin{description}[leftmargin=!]
	\item The degree of sensible heat of a body as measured by a thermometer or similar instrument.
\end{description}

\item[Transitional Attack] \hfill
\begin{description}[leftmargin=!]
	\item A Fire Attack where the application of water starts on the exterior and transitions to an interior application. 
	\item An offensive fire attack initiated by an exterior, indirect hand line operation into the fire compartment to initiate cooling while transitioning into interior direct fire attack in coordination with ventilation operations. 
\end{description}

\item[Under Control] \hfill
\begin{description}[leftmargin=!]
	\item A term used to describe when visible and audible signs of combustion are absent from the environment.
\end{description}

\item [Uni-directional vent] \hfill
\begin{description}[leftmargin=!]
	\item A building opening that serves as either an intake or an exhaust vent of a flow path at a given point in time.
\end{description}

\item [Vent] \hfill
\begin{description}[leftmargin=!]
	\item An opening for the passage of, or dissipation of, fluids, such as smoke, gases, and heat.
\end{description}

\item [Ventilation] \hfill
\begin{description}[leftmargin=!]
	\item Circulation of air in any space by natural wind or convection or by fans blowing air into or exhausting air out of a building; a fire-fighting operation of removing smoke, gases and heat from the structure by natural or mechanical methods.
\end{description}

\item[Ventilation Limited Fire] \hfill
\begin{description}[leftmargin=!]
	\item A fire in which the heat release rate or growth is controlled by the amount of air (oxygen) available to the fire. 
	\item (Ventilation Controlled) A fire with limited ventilation in which the heat-release rate or growth is limited by the amount of oxygen available to the fire.
	\item A fire in an enclosed building that is restricted because there is insufficient oxygen available for the fire to burn as rapidly as it would with an unlimited supply of oxygen. 
	\item A fire where every object in the fire compartment is fully involved in fire and the heat release rate depends on the airflow through the openings to the fire compartment.
\end{description}

\end{description}