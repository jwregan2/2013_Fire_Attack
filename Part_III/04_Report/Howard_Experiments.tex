\documentclass[12pt,oneside]{book}
\input{commoncommands}

%rename chapter headings
\renewcommand{\chaptername}{}
\renewcommand{\bibname}{References}

\usepackage{fancyhdr}
\pagestyle{fancy}
\lhead{}
\rhead{}
\chead{}
\renewcommand{\headrulewidth}{0pt}

\usepackage{draftwatermark}
\SetWatermarkText{DRAFT}
\SetWatermarkScale{1}

\begin{document}
\pagenumbering{gobble}
\bibliographystyle{unsrt}
	
\begin{minipage}[t][9in][s]{6.25in}

\headerB{
Impact of Fire Attack Utilizing \\
Interior and Exterior Streams on \\
Firefighter Safety and Occupant Survival: Howard Room Experiments \\
}

\normalsize

\headerC{
	{ \flushleft{
	Keith Stakes \\
	Robin Zevotek \\
	\vspace{0.2in}
	UL Firefighter Safety Research Institute \\
	Columbia, MD 20145 \\
	\vspace*{2\baselineskip}

	} 

	\vfill

	\flushright{

	\includegraphics[width=2.0in]{Figures/General/FSRI_GraphicShield.pdf} \\ [.3in]

	}
	}
	}

\end{minipage}

\newpage
\hspace{5in}
\newpage

\frontmatter

\begin{minipage}[t][9in][s]{6.25in}
\pagenumbering{gobble}


\headerB{
Impact of Fire Attack Utilizing \\
Interior and Exterior Streams on\\ 
Firefighter Safety and Occupant \\
Survival: Air Entrainment\\
}

\headerC{
\flushleft{
Keith Stakes \\
Robin Zevotek \\
\vspace{0.2in}
{UL Firefighter Safety Research Institute \\
Columbia, MD 21045 \\}}

\flushleft{\today \\}
}


\vfill

\flushright{\includegraphics[width=2in]{Figures/General/FSRI_GraphicShield.pdf}}

\titlesigs

\end{minipage}

\frontmatter

\paragraph{Experiment 1} \mbox{}

Experiment 1 was a room and contents fire in the bedroom of the structure testing the ability for the fire to regrow after an exterior attack. The door to the hallway and the bedroom window within the structure were open for the duration of the test. The fire was allowed to grow to steady state before suppression. Suppression was conducted from the exterior of the structure via a straight stream on a 150~gpm at 75~psi combination nozzle connected to a 1~3/4~in hoseline. The nozzle firefighter was positioned close to the window opening and the hose stream was directed at the ceiling via a max angle position. Water was applied for approximately 10 seconds. Figure \ref{fig:Exp1Config} shows the configuration of the structure and Table \ref{Table:Exp1Interventions} shows at what times interventions were performed. 

% The results of Experiment 1 can be found in Appendix \ref{App:Exp1Results}. To view the full experiment video \href{https://youtu.be/gl8rc1Nsl1k}{Click Here}.

\begin{figure}[H]
	\centering
	\includegraphics[width=5in]{Howard_Exp_1.png}
	\caption{Experiment 1 Configuration}
	\label{fig:Exp1Config}
\end{figure}

\begin{table}[H]
	\centering
	\caption{Experiment 1 Interventions}
	\begin{tabular}{|c|c|} 
		\hline
		Time & Intervention \\ \hline \hline
		00:00 & Ignition - Bedroom \\ \hline
		04:30 & Exterior Suppression \\ \hline
		13:00 & End Experiment\\ \hline
	\end{tabular}
	\label{Table:Exp1Interventions}
\end{table}

\clearpage

\paragraph{Experiment 2} \mbox{}

Experiment 2 was a room and contents fire in the bedroom of the structure testing the ability for the fire to regrow after an exterior attack. This test was similar to Experiment 1 with the exception of additional fuel loading in the fire room. The details of the differences in the fuel loading can be found in the Fuel Load section above. The door to the hallway and the bedroom window within the structure were open for the duration of the test. The fire was allowed to grow to steady state before suppression. Suppression was conducted from the exterior of the structure via a straight stream on a 150~gpm at 75~psi combination nozzle connected to a 1~3/4~in hoseline. The nozzle firefighter was positioned close to the window opening and the hose stream was directed at the ceiling via a max angle position. Water was applied for approximately 10 seconds. After 16 minutes, it was determined that the fire was not going to re-grow and thus was re-ignited manually in the bedroom and was allowed to grow uninhibited once again until a steady state condition was reached. Exterior suppression was completed as before with water flowing for approximately 8 seconds. After approximately 14 minutes, the fire re-grew to a new steady state and was suppressed one last time using the same method as above with water flowing for approximately 10 seconds. Figure \ref{fig:Exp2Config} shows the configuration of the structure and Table \ref{Table:Exp2Interventions} shows at what times interventions were performed.  

% The results of Experiment 1 can be found in Appendix \ref{App:Exp1Results}. To view the full experiment video \href{https://youtu.be/gl8rc1Nsl1k}{Click Here}.

\begin{figure}[H]
	\centering
	\includegraphics[width=5in]{Howard_Exp_2.png}
	\caption{Experiment 2 Configuration}
	\label{fig:Exp2Config}
\end{figure}

\begin{table}[H]
	\centering
	\caption{Experiment 2 Interventions}
	\begin{tabular}{|c|c|} 
		\hline
		Time & Intervention \\ \hline \hline
		00:00 & Ignition - Bedroom \\ \hline
		05:00 & Exterior Suppression \\ \hline
		21:00 & Second Ignition - Bedroom \\ \hline
		27:00 & Exterior Suppression \\ \hline
		40:20 & Exterior Suppression \\ \hline
		44:00 & End Experiment\\ \hline
	\end{tabular}
	\label{Table:Exp2Interventions}
\end{table}

\clearpage

\paragraph{Experiment 3} \mbox{}

Experiment 3 was a room and contents fire in the bedroom of the structure testing the impact of hose stream air entrainment on fire behavior and suppression capability. The door to the hallway and the bedroom window within the structure were open for the duration of the test. The fire was allowed to grow to steady state before suppression. Suppression was conducted from the interior of the structure via a straight stream on a 150~gpm at 75~psi combination nozzle connected to a 1~3/4~in hoseline. The nozzle firefighter advanced down the hallway and the hose stream was directed ahead in a wall-ceiling-wall pattern before entering the fire room for final extinguishment. Figure \ref{fig:Exp3Config} shows the configuration of the structure and Table \ref{Table:Exp3Interventions} shows at what times interventions were performed.  

% The results of Experiment 1 can be found in Appendix \ref{App:Exp1Results}. To view the full experiment video \href{https://youtu.be/gl8rc1Nsl1k}{Click Here}.

\begin{figure}[H]
	\centering
	\includegraphics[width=5in]{Howard_Exp_3.png}
	\caption{Experiment 3 Configuration}
	\label{fig:Exp3Config}
\end{figure}

\begin{table}[H]
	\centering
	\caption{Experiment 3 Interventions}
	\begin{tabular}{|c|c|} 
		\hline
		Time & Intervention \\ \hline \hline
		00:00 & Ignition - Bedroom \\ \hline
		05:00 & Interior Suppression \\ \hline
		15:00 & End Experiment\\ \hline
	\end{tabular}
	\label{Table:Exp3Interventions}
\end{table}

\clearpage

\paragraph{Experiment 4} \mbox{}

Experiment 4 was a room and contents fire in the bedroom of the structure testing the impact of hose stream air entrainment on fire behavior and suppression capability. The door to the hallway and the bedroom window within the structure were open for the duration of the test. The fire was allowed to grow to steady state before suppression. Suppression was conducted from the interior of the structure via a narrow fog stream on a 150~gpm at 75~psi combination nozzle connected to a 1~3/4~in hoseline. The nozzle firefighter advanced down the hallway and the hose stream was directed ahead in an `O' pattern before entering the fire room for final extinguishment. Figure \ref{fig:Exp4Config} shows the configuration of the structure and Table \ref{Table:Exp4Interventions} shows at what times interventions were performed. 

% The results of Experiment 1 can be found in Appendix \ref{App:Exp1Results}. To view the full experiment video \href{https://youtu.be/gl8rc1Nsl1k}{Click Here}.

\begin{figure}[H]
	\centering
	\includegraphics[width=5in]{Howard_Exp_4.png}
	\caption{Experiment 4 Configuration}
	\label{fig:Exp4Config}
\end{figure}

\begin{table}[H]
	\centering
	\caption{Experiment 4 Interventions}
	\begin{tabular}{|c|c|} 
		\hline
		Time & Intervention \\ \hline \hline
		00:00 & Ignition - Bedroom \\ \hline
		07:30 & Interior Suppression \\ \hline
		14:00 & End Experiment\\ \hline
	\end{tabular}
	\label{Table:Exp4Interventions}
\end{table}

\clearpage

\paragraph{Experiment 5} \mbox{}

Experiment 5 was a room and contents fire in the bedroom of the structure testing the impact of door control. The bedroom window within the structure was open for the duration of the test. The fire was allowed to grow to steady state before the hall door was manipulated from the exterior of the structure. Several iterations of door open and door closed were performed before suppression. Suppression was conducted from the exterior of the structure via a straight stream on a 150~gpm at 75~psi combination nozzle connected to a 1~3/4~in hoseline. The nozzle firefighter was positioned close to the window opening and the hose stream was directed at the ceiling via a max angle position. Water was applied for approximately 10 seconds. Figure \ref{fig:Exp5Config} shows the configuration of the structure and Table \ref{Table:Exp5Interventions} shows at what times interventions were performed.

% The results of Experiment 1 can be found in Appendix \ref{App:Exp1Results}. To view the full experiment video \href{https://youtu.be/gl8rc1Nsl1k}{Click Here}.

\begin{figure}[H]
	\centering
	\includegraphics[width=5in]{Howard_Exp_5.png}
	\caption{Experiment 5 Configuration}
	\label{fig:Exp5Config}
\end{figure}

\begin{table}[H]
	\centering
	\caption{Experiment 5 Interventions}
	\begin{tabular}{|c|c|} 
		\hline
		Time & Intervention \\ \hline \hline
		00:00 & Ignition - Bedroom \\ \hline
		04:30 & Hall Door Closed \\ \hline
		05:00 & Hall Door Open \\ \hline
		16:00 & Hall Door Closed \\ \hline
		07:00 & Hall Door Open \\ \hline
		08:00 & Hall Door Closed \\ \hline
		09:00 & Hall Door Open \\ \hline
		11:00 & Exterior Suppression \\ \hline
		20:00 & End Experiment\\ \hline
	\end{tabular}
	\label{Table:Exp5Interventions}
\end{table}

\clearpage

\paragraph{Experiment 6} \mbox{}

Experiment 6 was a room and contents fire in the bedroom of the structure testing the impact of door control. The bedroom window within the structure was closed for the duration of the test. The fire was allowed to grow to steady state before the hall door was manipulated from the exterior of the structure. Several iterations of door open and door closed were performed before the bedroom window was ventilated from the exterior. Additional manipulations of the hall door followed ventilation and before interior suppression was conducted. Suppression was conducted from the interior of the structure via a straight stream on a 150~gpm at 75~psi combination nozzle connected to a 1~3/4~in hoseline. The nozzle firefighter advanced down the hallway and the hose stream was directed ahead in a wall-ceiling-wall pattern before entering the fire room for final extinguishment. Figure \ref{fig:Exp6Config} shows the configuration of the structure and Table \ref{Table:Exp6Interventions} shows at what times interventions were performed.

% The results of Experiment 1 can be found in Appendix \ref{App:Exp1Results}. To view the full experiment video \href{https://youtu.be/gl8rc1Nsl1k}{Click Here}.

\begin{figure}[H]
	\centering
	\includegraphics[width=5in]{Howard_Exp_6.png}
	\caption{Experiment 6 Configuration}
	\label{fig:Exp6Config}
\end{figure}

\begin{table}[H]
	\centering
	\caption{Experiment 6 Interventions}
	\begin{tabular}{|c|c|} 
		\hline
		Time & Intervention \\ \hline \hline
		00:00 & Ignition - Bedroom \\ \hline
		04:00 & Hall Door Closed \\ \hline
		04:30 & Hall Door Open \\ \hline
		05:30 & Hall Door Closed \\ \hline
		06:30 & Hall Door Open \\ \hline
		11:30 & Second Ignition - Bedroom \\ \hline
		16:45 & Hall Door Closed \\ \hline
		17:15 & Hall Door Open \\ \hline
		19:45 & Hall Door Closed \\ \hline
		20:15 & Hall Door Open \\ \hline
		21:00 & Bedroom Window Vent \\ \hline
		23:30 & Interior Suppression \\ \hline
		35:00 & End Experiment\\ \hline
	\end{tabular}
	\label{Table:Exp6Interventions}
\end{table}

\clearpage

\end{document}